% generated by GAPDoc2LaTeX from XML source (Frank Luebeck)
\documentclass[a4paper,11pt]{report}

\usepackage[top=37mm,bottom=37mm,left=27mm,right=27mm]{geometry}
\sloppy
\pagestyle{myheadings}
\usepackage{amssymb}
\usepackage[latin1]{inputenc}
\usepackage{makeidx}
\makeindex
\usepackage{color}
\definecolor{FireBrick}{rgb}{0.5812,0.0074,0.0083}
\definecolor{RoyalBlue}{rgb}{0.0236,0.0894,0.6179}
\definecolor{RoyalGreen}{rgb}{0.0236,0.6179,0.0894}
\definecolor{RoyalRed}{rgb}{0.6179,0.0236,0.0894}
\definecolor{LightBlue}{rgb}{0.8544,0.9511,1.0000}
\definecolor{Black}{rgb}{0.0,0.0,0.0}

\definecolor{linkColor}{rgb}{0.0,0.0,0.554}
\definecolor{citeColor}{rgb}{0.0,0.0,0.554}
\definecolor{fileColor}{rgb}{0.0,0.0,0.554}
\definecolor{urlColor}{rgb}{0.0,0.0,0.554}
\definecolor{promptColor}{rgb}{0.0,0.0,0.589}
\definecolor{brkpromptColor}{rgb}{0.589,0.0,0.0}
\definecolor{gapinputColor}{rgb}{0.589,0.0,0.0}
\definecolor{gapoutputColor}{rgb}{0.0,0.0,0.0}

%%  for a long time these were red and blue by default,
%%  now black, but keep variables to overwrite
\definecolor{FuncColor}{rgb}{0.0,0.0,0.0}
%% strange name because of pdflatex bug:
\definecolor{Chapter }{rgb}{0.0,0.0,0.0}
\definecolor{DarkOlive}{rgb}{0.1047,0.2412,0.0064}


\usepackage{fancyvrb}

\usepackage{mathptmx,helvet}
\usepackage[T1]{fontenc}
\usepackage{textcomp}
\usepackage{amsmath}


\usepackage[
            pdftex=true,
            bookmarks=true,        
            a4paper=true,
            pdftitle={Written with GAPDoc},
            pdfcreator={LaTeX with hyperref package / GAPDoc},
            colorlinks=true,
            backref=page,
            breaklinks=true,
            linkcolor=linkColor,
            citecolor=citeColor,
            filecolor=fileColor,
            urlcolor=urlColor,
            pdfpagemode={UseNone}, 
           ]{hyperref}

\newcommand{\maintitlesize}{\fontsize{50}{55}\selectfont}

% write page numbers to a .pnr log file for online help
\newwrite\pagenrlog
\immediate\openout\pagenrlog =\jobname.pnr
\immediate\write\pagenrlog{PAGENRS := [}
\newcommand{\logpage}[1]{\protect\write\pagenrlog{#1, \thepage,}}
%% were never documented, give conflicts with some additional packages

\newcommand{\GAP}{\textsf{GAP}}

%% nicer description environments, allows long labels
\usepackage{enumitem}
\setdescription{style=nextline}

%% depth of toc
\setcounter{tocdepth}{1}





%% command for ColorPrompt style examples
\newcommand{\gapprompt}[1]{\color{promptColor}{\bfseries #1}}
\newcommand{\gapbrkprompt}[1]{\color{brkpromptColor}{\bfseries #1}}
\newcommand{\gapinput}[1]{\color{gapinputColor}{#1}}


\begin{document}

\logpage{[ 0, 0, 0 ]}
\begin{titlepage}
\mbox{}\vfill

\begin{center}{\maintitlesize \textbf{The \textsf{HERmitian} Package\mbox{}}}\\
\vfill

\hypersetup{pdftitle=The \textsf{HERmitian} Package}
\markright{\scriptsize \mbox{}\hfill The \textsf{HERmitian} Package \hfill\mbox{}}
{\Huge \textbf{Divisors and Riemann-Roch Spaces of Algebraic Function Fields of Hermitian
Curves\mbox{}}}\\
\vfill

{\Huge Version 0.1\mbox{}}\\[1cm]
{14 March 2019\mbox{}}\\[1cm]
\mbox{}\\[2cm]
{\Large \textbf{ G{\a'a}bor P. Nagy   \mbox{}}}\\
{\Large \textbf{ Sabira El Khalfaoui  \mbox{}}}\\
\hypersetup{pdfauthor= G{\a'a}bor P. Nagy   ;  Sabira El Khalfaoui  }
\end{center}\vfill

\mbox{}\\
{\mbox{}\\
\small \noindent \textbf{ G{\a'a}bor P. Nagy   }  Email: \href{mailto://nagyg@math.u-szeged.hu} {\texttt{nagyg@math.u-szeged.hu}}\\
  Homepage: \href{http://www.math.u-szeged.hu/~nagyg/} {\texttt{http://www.math.u-szeged.hu/\texttt{\symbol{126}}nagyg/}}}\\
{\mbox{}\\
\small \noindent \textbf{ Sabira El Khalfaoui  }  Email: \href{mailto://sabira@math.u-szeged.hu} {\texttt{sabira@math.u-szeged.hu}}}\\
\end{titlepage}

\newpage\setcounter{page}{2}
{\small 
\section*{Copyright}
\logpage{[ 0, 0, 1 ]}
 \index{License} {\copyright} 2019 by G{\a'a}bor P. Nagy

 \textsf{HERmitian} package is free software; you can redistribute it and/or modify it under the
terms of the \href{http://www.fsf.org/licenses/gpl.html} {GNU General Public License} as published by the Free Software Foundation; either version 2 of the License,
or (at your option) any later version. \mbox{}}\\[1cm]
{\small 
\section*{Acknowledgements}
\logpage{[ 0, 0, 2 ]}
 We appreciate very much all past and future comments, suggestions and
contributions to this package and its documentation provided by \textsf{GAP} users and developers. \mbox{}}\\[1cm]
\newpage

\def\contentsname{Contents\logpage{[ 0, 0, 3 ]}}

\tableofcontents
\newpage

     
\chapter{\textcolor{Chapter }{Introduction}}\label{HERmitian Introduction}
\logpage{[ 1, 0, 0 ]}
\hyperdef{L}{X7DFB63A97E67C0A1}{}
{
  \index{HERmitian package} This chapter describes the \textsf{GAP} package \textsf{HERmitian}. This package implements functionalities for divisors and Riemann-Roch spaces
of an algebraic function field of Hermitian. 

 If you are viewing this with on-line help, type: 

 
\begin{Verbatim}[commandchars=!@|,fontsize=\small,frame=single,label=Example]
  !gapprompt@gap>| !gapinput@?HERmitian package|
\end{Verbatim}
 

 to see the functions provided by the \textsf{HERmitian} package.

  
\section{\textcolor{Chapter }{Unpacking the \textsf{HERmitian} Package}}\label{Unpacking the HERmitian Package}
\logpage{[ 1, 1, 0 ]}
\hyperdef{L}{X85E7A5BB7BF23319}{}
{
  If the \textsf{HERmitian} package was obtained as a part of the \textsf{GAP} distribution from the ``Download'' section of the \textsf{GAP} website, you may proceed to Section \ref{Compiling Binaries of the HERmitian Package}. Alternatively, the \textsf{HERmitian} package may be installed using a separate archive, for example, for an update
or an installation in a non-default location (see  (\textbf{Reference: GAP Root Directories})). 

 Below we describe the installation procedure for the \texttt{.tar.gz} archive format. Installation using other archive formats is performed in a
similar way. 

 To install the \textsf{HERmitian} package, unpack the archive file, which should have a name of form \texttt{HERmitian-\mbox{\texttt{\mdseries\slshape XXX}}.tar.gz} for some version number \mbox{\texttt{\mdseries\slshape XXX}}, by typing 

 {\nobreakspace}{\nobreakspace}\texttt{gzip -dc HERmitian-\mbox{\texttt{\mdseries\slshape XXX}}.tar.gz | tar xpv} 

 It may be unpacked in one of the following locations: 
\begin{itemize}
\item  in the \texttt{pkg} directory of your \textsf{GAP}{\nobreakspace}4 installation; 
\item  or in a directory named \texttt{.gap/pkg} in your home directory (to be added to the \textsf{GAP} root directory unless \textsf{GAP} is started with \texttt{-r} option); 
\item  or in a directory named \texttt{pkg} in another directory of your choice (e.g.{\nobreakspace}in the directory \texttt{mygap} in your home directory). 
\end{itemize}
 In the latter case one one must start \textsf{GAP} with the \texttt{-l} option, e.g.{\nobreakspace}if your private \texttt{pkg} directory is a subdirectory of \texttt{mygap} in your home directory you might type: 

 {\nobreakspace}{\nobreakspace}\texttt{gap -l ";\mbox{\texttt{\mdseries\slshape myhomedir}}/mygap"} 

 where \mbox{\texttt{\mdseries\slshape myhomedir}} is the path to your home directory, which (since \textsf{GAP}{\nobreakspace}4.3) may be replaced by a tilde (the empty path before the
semicolon is filled in by the default path of the \textsf{GAP}{\nobreakspace}4 home directory). }

  
\section{\textcolor{Chapter }{Loading the \textsf{HERmitian} Package}}\label{Loading the HERmitian Package}
\logpage{[ 1, 2, 0 ]}
\hyperdef{L}{X84A8B0C583898BB8}{}
{
  To use the \textsf{HERmitian} Package you have to request it explicitly. This is done by calling \texttt{LoadPackage} (\textbf{Reference: LoadPackage}): 

 
\begin{Verbatim}[commandchars=!@|,fontsize=\small,frame=single,label=Example]
  !gapprompt@gap>| !gapinput@LoadPackage("HERmitian");|
  ----------------------------------------------------------------
  Loading  HERmitian 0.1
  by G�bor P. Nagy (http://www.math.u-szeged.hu/~nagyg)
  For help, type: ?HERmitian package 
  ----------------------------------------------------------------
  true
\end{Verbatim}
 

 If \textsf{GAP} cannot find a working binary, the call to \texttt{LoadPackage} will still succeed but a warning is issued informing that the \texttt{HelloWorld()} function will be unavailable. 

 If you want to load the \textsf{HERmitian} package by default, you can put the \texttt{LoadPackage} command into your \texttt{gaprc} file (see Section{\nobreakspace} (\textbf{Reference: The gap.ini and gaprc files})). }

  
\section{\textcolor{Chapter }{Testing the \textsf{HERmitian} Package}}\label{Loading the HERmitian Package}
\logpage{[ 1, 3, 0 ]}
\hyperdef{L}{X8456590879EA0EBD}{}
{
  You can run tests for the package by 
\begin{Verbatim}[commandchars=!@|,fontsize=\small,frame=single,label=Example]
  !gapprompt@gap>| !gapinput@Test(Filename(DirectoriesPackageLibrary("HERmitian"),"../tst/testall.tst"));|
\end{Verbatim}
 }

 }

        
\chapter{\textcolor{Chapter }{Mathematical background}}\label{HERmitian Background}
\logpage{[ 2, 0, 0 ]}
\hyperdef{L}{X7EF1B6708069B0C7}{}
{
   Our notation and terminology are standard. The reader is referred to \cite{HKT_book}, \cite{Stichtenoth_book}. For the decoding of algebraic-geometric codes see the survey paper \cite{HoholdtPellikaan}. 
\section{\textcolor{Chapter }{Algebraic curves, places, divisors}}\label{curves_places_divisors}
\logpage{[ 2, 1, 0 ]}
\hyperdef{L}{X87F7152A83FFE016}{}
{
  

An algebraic plane curve $\mathscr{X}$ over the field $K$ is given by a polynomial $f(X,Y)\in K[X,Y]$ of degree $n$; the usual notation is $\mathscr{X}:f(X,Y)=0$. The \emph{affine points} of $\mathscr{X}$ are pairs $(x,y)\in L^2$, where $L$ is an extension field of $K$ and $f(x,y)=0$ holds. We say that $(x,y)$ is a \emph{smooth point} of $\mathscr{X}$ if $(\frac{\partial{f}}{\partial{X}}(x,y), \frac{\partial{f}}{\partial{Y}}(x,y))
\neq (0,0)$. At a smooth affine point $(x,y)\in L^2$, the curve has formal local parametrization $(\xi(t),\eta(t))\in L[[t]]^2$ such that $\xi(0)=x$, $\eta(0)=y$ and $f(\xi(t),\eta(t))=0$. Non smooth points are called \emph{singular.} 

The affine curve $\mathscr{X}:f(X,Y)=0$ has \emph{homogeneous equation} $F(X,Y,Z)=0$ with $F(X,Y,Z)=Z^nf(\frac{X}{Z},\frac{Y}{Z})$. The \emph{projective points} of $\mathscr{X}$ satisfy $F(x,y,z)=0$. In particular, the affine point $(x,y)$ of $\mathcal{X}$ corresponds to a projective point $(x:y:1)$. The points of $\mathscr{X}$ at infinity are given by the homogeneous equation $F(X,Y,0)=0$. Smoothness and local parametrization at projective points are defined in the
obvious way. We say that the projective point $(x:y:z)$ of $\mathscr{X}$ is \emph{defined over $L$} if $x/y,y/z,z/x$ are either infinite or in $L$. Notice that any singular point (affine or projective) is defined over an
algebraic extension of the underlying field $K$. 

The algebraic curve $\mathscr{X}$ is said to be \emph{nonsingular} or \emph{smooth,} if all its points are smooth. This implies that $f$ is absolutely irreducible. For smooth algebraic plane curves, the concept of a \emph{place} is equivalent with the concept of a point, when $\mathscr{X}$ is considered as a curve over the algebraic closure of $K$. A \emph{divisor} is a formal sum $D=n_1P_1+\ldots+n_kP_k$ with integers $n_1,\ldots,n_k$ and places $P_1,\ldots,P_k$. The degree of $D$ is $n_1+\cdots+n_k$. The integer $n_i$ is the \emph{valuation} $v_{P_i}(D)$ of $D$ at $P_i$; for $P\neq P_i$ one has $v_P(D)=0$. The \emph{support} of $D$ is the set of places $P$ such that $v_P(D)\neq 0$. }

 
\section{\textcolor{Chapter }{Function fields and Riemann-Roch spaces}}\label{riemann_roch_spaces}
\logpage{[ 2, 2, 0 ]}
\hyperdef{L}{X79D81F6581FEE07D}{}
{
  

Let $\mathscr{X}:f(X,Y)=0$ be a smooth plane algebraic curve. The function field $K(\mathscr{X})$ of $\mathscr{X}$ is generated by the variables $x,y$ subject to the algebraic relation $f(x,y)=0$. In particular, each element of $K(\mathscr{X})$ can be written as $a(x,y)/b(x,y)$ with $a,b\in K[X,Y]$. Let $h\in K(\mathscr{X})$ and a place $P$ of $\mathscr{X}$, we define the valuation $v_P(h)$ as the subdegree of $h(\xi(t),\eta(t))$, where $(\xi(t),\eta(t))$ is the formal local parametrization at $P$. If $v_P(h)>0$ then $P$ is a \emph{zero} of $h$, if $v_P(h)<0$ then $P$ is a \emph{pole} of $h$. If $v_P(h)\geq 0$, then $h(P)=h(\xi(0),\eta(0))$ is a well-defined element of $K$. 

For every non-zero function $h\in K(\mathscr{X})$, $\mathrm{Div}(h)$ stands for the principal divisor associated with $h$ while $\mathrm{Div}(h)_0$ and $\mathrm{Div}(h)_\infty$ for its zero and pole divisor. Furthermore, for every separable function $h\in K (\mathscr{X})$, $dh$ is the exact differential arising from $h$, and $\Omega$ denotes the set of all these differentials. Also, ${\mathrm{res}}_P(dh)$ is the residue of $dh$ at a place of $P$ of $K(\mathscr{X})$. 

For any divisor $A$ of $K(\mathscr{X})$, the \emph{Riemann-Roch space} of $A$ is 
\[\mathcal{L}(A)=\{h\in K(\mathscr{X})\setminus \{0\}|\, \mathrm{Div}(h)\succeq
-A \}\cup \{0\}.\]
 We denote $\ell(A)=\dim(\mathcal{L}(A))$. Furthermore, the \emph{differential space} of $A$ is 
\[\Omega(A)=\{dh\in \Omega \mid \mathrm{Div}(dh)\succeq A \}\cup \{0\}.\]
 Both the Riemann-Roch and the differential spaces are linear spaces over $K$. Their dimensions are given by the theorem of Riemann-Roch: 
\[\ell(A)=\deg(A)+1-g+\ell(W-A).\]
 Here, $W$ is a canonical divisor of $\mathscr{X}$, and $g$ is the \emph{genus} of $\mathscr{X}$. The latter is the most important birational invariant of an algebraic curve.
For smooth curves of degree $n$, the genus formula is 
\[g=\frac{(n-1)(n-2)}{2}.\]
 The theorem of Riemann-Roch implies 
\[\ell(A)\geq \deg(A)+1-g,\]
 with equality if $\deg(A)>2g-2$. }

 
\section{\textcolor{Chapter }{Automorphisms of algebraic curves}}\label{automorphisms_curves}
\logpage{[ 2, 3, 0 ]}
\hyperdef{L}{X875FAF2D817EB7DB}{}
{
  

Let $\mathscr{X}:f(X,Y)=0$ be a smooth plane algebraic curve with function field $K(\mathscr{X})=K(x,y),$ where the elements $x,y$ are subject to the algebraic relation $f(x,y)=0$. We assume that $K$ is the constant field of $K(\mathscr{X})$. An \emph{automorphism} of $\mathscr{X}$ is an automorphism of the function field, leaving all elements of $K$ fixed. In particular, for any automorphism $\alpha$ of $\mathscr{X}$, there are polynomials $u,v,w\in K[X,Y]$ such that 
\[\alpha:(x,y) \to \left(\frac{u(x,y)}{w(x,y)},\frac{v(x,y)}{w(x,y)}\right).\]
 Substituting formal power series in $\alpha$, we obtain an action of $\alpha$ on the set of places of $\mathscr{X}$. This extends to an action on divisors, differentials and Riemann-Roch
spaces. }

 
\section{\textcolor{Chapter }{Algebraic plane curves over finite fields}}\label{curves_finite_fields}
\logpage{[ 2, 4, 0 ]}
\hyperdef{L}{X7BDEE9C27CA403BA}{}
{
  

Let $p$ be a prime and $K$ an algebraically closed field of characteristic $p$. For $q=p^e$ we define the \emph{Frobenius automorphism} $\mathrm{Frob}_q:x\mapsto x^q$ of $K$. This extends to an Frobenius map of $K$-polynomials (acting on the coefficients) and of affine and projective points
over $K$ (acting on the coordinates). The curve $\mathscr{X}$ is said to be $\mathbb{F}_q$-rational, if it is $\mathrm{Frob}_q$-invariant. Moreover, the Frobenius action extends to places and divisors of $\mathbb{F}_q$-rational curves, which allows us to speak of places and divisors defined over $\mathbb{F}_q$. Let $\mathscr{X}$ be an algebraic plane curve over $\mathbb{F}_q$ and $P$ a place of $\mathscr{X}$. Let $r$ be the smallest positive integer such that $P$ is defined over $\mathbb{F}_{q^r}$. Then, the divisor 
\[P+P^{\mathrm{Frob}_q}+P^{\mathrm{Frob}_q^2}+\cdots+P^{\mathrm{Frob}_q^{r-1}}\]
 is an $\mathbb{F}_q$-rational place of degree $r$ of $\mathscr{X}$. 

If $A$ is an $\mathbb{F}_q$-rational divisor then the Riemann-Roch space $\mathcal{L}(A)$ has a basis which consists of $\mathbb{F}_q$-rational elements of the function field of $\mathscr{X}$. Hence, we can view $\mathcal{L}(A)$ as an $\mathbb{F}_q$-linear space of dimension $\ell(A)$. Similarly, $\Omega(A)$ can be seen as a vector space over $\mathbb{F}_q$. 

If $\mathscr{X}$ is an algebraic curve over $\mathbb{F}_q$ and $\alpha$ is an automorphism of $\mathscr{X}$, then we say that $\alpha$ is defined over $\mathbb{F}_q$ provided $\alpha$ commutes with the Frobenius map $\mathrm{Frob}_q$. The automorphisms of $\mathscr{X}$ which are defined over $\mathbb{F}_q$ form a subgroup of $\mathrm{Aut}(\mathscr{X})$. }

 
\section{\textcolor{Chapter }{Algebraic-geometry codes}}\label{ag_codes}
\logpage{[ 2, 5, 0 ]}
\hyperdef{L}{X85C45F447B0F7202}{}
{
  

Algebraic-geometry (AG) codes are linear codes constructed from algebraic
curves defined over a finite field $\mathbb{F}_{q}$. The best known such general construction was originally introduced by Goppa,
see \cite{Goppa}. It provides linear codes from certain rational functions whose poles are
prescribed by a given $\mathbb{F}_{q}$-rational divisor $G$, by evaluating them at some set of $\mathbb{F}_{q}$-rational places disjoint from $\mathrm{supp}(G)$. The dual to such a code can be obtained by computing residues of
differential forms. The former are the \emph{functional} codes, and the latter are the \emph{differential} codes. 

Let $\mathscr{X}$ be a smooth plane curve defined over the finite field $\mathbb{F}_q$. Write $D=Q_1+\ldots+Q_n$ for the $\mathbb{F}_{q}$-rational places $Q_1,\ldots,Q_n$. Let $G$ be another divisor of $\mathbb{F}_{q}(\mathscr{X})$ whose support $\mathrm{supp}(G)$ contains none of the places $Q_i$ with $1\le i \le n$. For any function $h\in \mathcal{L}(G)$, the \emph{evaluation} of $h$ at $D$ is given by 
\[\mathrm{ev}_D(h)=(h(Q_1),\ldots h(Q_n)).\]
 This defines the \emph{evaluation map} $\mathrm{ev}_D:\mathcal{L}(G)\to \mathbb{F}_{q}^n$ which is $\mathbb{F}_{q}$-linear and also injective when $n>\deg(G)$. Therefore, its image is a subspace of the vector space $\mathbb{F}_{q}^n$, or equivalently, an AG $[n,k,d]$-code where $d\geq n-\deg(G)$ and if $\deg(G)>2g-2$ then $k= \deg(G)+1-g$. Such a code is the \emph{functional} code 
\[C_L(D,G) = \{ (h(Q_1),\ldots,h(Q_n)) \mid h\in \mathcal{L}(G)\}\]
 with \emph{designed minimum distance} $ n-\deg(G)$. The dual code 
\[C_\Omega(D,G) = \{ (\mathrm{res}_{Q_1}(dh), \ldots, \mathrm{res}_{Q_1}(dh))
\mid dh\in\Omega(G-D)\}\]
 of $C_L(D,G)$ is named a \emph{differential code}. The differential code $C_\Omega(D,G)$ is a $[n,\ell(G-D)-\ell(G)+\deg D,d]$-code with $d\geq \deg(G)-(2g-2)$, and its designed minimum distance is $\deg(G)-(2g-2)$. 

Typically the divisor $G$ is taken to be a multiple $mP$ of a single place $P$ of degree one. Such codes are the \emph{one-point} codes, and have been extensively investigated. It has been shown however that
AG-codes with better parameters than the comparable one-point Hermitian code
may be obtained by allowing the divisor $G$ to be more general, see \cite{MatthewsMichel} and the references therein. }

 
\section{\textcolor{Chapter }{Hermitian curves over finite fields}}\label{hermitian_curves}
\logpage{[ 2, 6, 0 ]}
\hyperdef{L}{X7AB6DDB382EBEFAD}{}
{
  

This package implements places, divisors and Riemann-Roch spaces of the \emph{Hermitian curve} $\mathscr{H}_q$ defined over $\mathbb{F}_{q^2}$. We quote the most important geometric and combinatorial properties of $\mathscr{H}_q$, the references are \cite{Hirschfeld} and \cite{HughesPiper}. In the projective plane $PG(2,\mathbb{F}_{q^2})$ equipped with homogeneous coordinates $(X:Y:Z)$, a canonical form of $\mathscr{H}_q$ is $X^{q+1}-Y^qZ-YZ^q=0$ so that 
\[\mathscr{H}_q:X^{q+1}=Y^q+Y\]
 in the affine equation. Every $\mathbb{F}_{q^2}$-rational place of the function field $\mathbb{F}_{q^2}(\mathscr{H}_q)$ of $\mathscr{H}_q$ corresponds to a point of $\mathscr{H}_q$ in $PG(2,\mathbb{F}_{q^2})$, and this holds true for the degree one places of the constant field
extension $\mathbb{F}_{q^{2k}}(\mathscr{H}_q)$ which correspond to the points of $\mathscr{H}_q$ in $PG(2,\mathbb{F}_{q^{2k}})$. Moreover, a place $P$ of degree $r>1$ of $\mathbb{F}_{q^2}(\mathscr{H}_q)$ is represented by a divisor $P_1+P_2+\ldots +P_r$ of the constant field extension $\mathbb{F}_{q^{2r}}(\mathscr{H}_q)$ where $P_i$ are degree one places of $\mathbb{F}_{q^{2r}}(\mathscr{H}_q)$ with $P_i=P_1^{\mathrm{Frob}_{q^2}^i}$ for $i=0,1,\ldots,r-1$. Furthermore, 
\[|\mathscr{H}_q(\mathbb{F}_{q^2})|=|\mathscr{H}_q(\mathbb{F}_{q^4})|=q^3+1\]
 and 
\[|\mathscr{H}_q(\mathbb{F}_{q^6})|=q^6+1+q^4(q-1),\]
 where $\mathscr{H}_q(K)$ denotes the set of $K$-rational points of the projective curve $\mathscr{H}_q$. A line $l$ of $PG(2, \mathbb{F}_{q^2})$ is either a tangent to $\mathscr{H}_q$ at an ${\mathbb{F}}_{q^2}$-rational point of $\mathscr{H}_q$ or it meets $\mathscr{H}_q$ at $q+1$ distinct ${\mathbb{F}}_{q^2}$-rational points. In terms of intersection divisors, see
\texttt{\symbol{92}}cite[Section
6.2]\texttt{\symbol{123}}HKT{\textunderscore}book\texttt{\symbol{125}}, 
\[ I(\mathscr{H}_q,l)=(q+1)Q \]
 for the point $Q\in \mathscr{H}_q({\mathbb{F}}_{q^2})$ of tangent $l$ of $\mathscr{H}_q$, and 
\[I(\mathscr{H}_q,l)=\sum_{i=1}^{q+1} Q_i\]
 for the $q+1$ distinct points of intersections $Q_1,\ldots,Q_{q+1}$ of $l$ and $\mathscr{H}_q$. 

Through every point $V\in PG(2,{\mathbb{F}}_{q^2})$ not in $\mathscr{H}_q({\mathbb{F}}_{q^2})$ there are $q^2-q+1$ secants and $q+1$ tangents to $\mathscr{H}_q$. The corresponding $q+1$ tangency points are the common points of $\mathscr{H}_q$ with the polar line of $V$ relative to the unitary polarity associated to $\mathscr{H}_q$. Let $V=(1:0:0)$. Then the line $l_\infty$ of equation $Z=0$ is tangent at $P_\infty=(0:1:0)$ while another line through $V$ with equation $Y-cZ=0$ is either a tangent or a secant according as $c^q+c$ is $0$ or not. 

If $K$ is the algebraic closure of $\mathbb{F}_{q^2}$ with $q>2$, then the group of $K$-automorphisms of the Hermitian curve $\mathscr{H}_q$ is the projective unitary group $PGU(3,q)$. In particular, all automorphisms of $\mathscr{H}_q$ are defined over $\mathbb{F}_{q^2}$. The automorphism group act doubly transitively on the set of $\mathbb{F}_{q^2}$-rational points.  }

 }

        
\chapter{\textcolor{Chapter }{How to use the package}}\label{HERmitian Usage}
\logpage{[ 3, 0, 0 ]}
\hyperdef{L}{X7FC3D36E8639C331}{}
{
   
\section{\textcolor{Chapter }{Hermitian curves}}\label{Curves}
\logpage{[ 3, 1, 0 ]}
\hyperdef{L}{X877AA77B8442240B}{}
{
  The following functions are available: 

\subsection{\textcolor{Chapter }{IsHermitian{\textunderscore}Curve}}
\logpage{[ 3, 1, 1 ]}\nobreak
\hyperdef{L}{X85A5A3998595C0A0}{}
{\noindent\textcolor{FuncColor}{$\triangleright$\enspace\texttt{IsHermitian{\textunderscore}Curve({\mdseries\slshape obj})\index{IsHermitianuScoreCurve@\texttt{IsHermitian{\textunderscore}Curve}}
\label{IsHermitianuScoreCurve}
}\hfill{\scriptsize (Category)}}\\


 Hermitian curve $H(q)$ is an algebraic curve over an algebraically closed field, having an affine
equation $X^{q+1} = Y^q + Y$. The base field of $H(q)$ is $GF(q^2)$. }

 

\subsection{\textcolor{Chapter }{Hermitian{\textunderscore}Curve}}
\logpage{[ 3, 1, 2 ]}\nobreak
\hyperdef{L}{X857339367E2063F3}{}
{\noindent\textcolor{FuncColor}{$\triangleright$\enspace\texttt{Hermitian{\textunderscore}Curve({\mdseries\slshape K, hratfn})\index{HermitianuScoreCurve@\texttt{Hermitian{\textunderscore}Curve}}
\label{HermitianuScoreCurve}
}\hfill{\scriptsize (operation)}}\\


 returns the corresponding Hermitian curve $H(q)$ over the algebraic closure of the field \mbox{\texttt{\mdseries\slshape K}}. The indeterminates $X,Y$ of \mbox{\texttt{\mdseries\slshape hratfn}} generate the corresponding Hermitian function field $K(X,Y)$ such that $X^{q+1} = Y^q + Y$. \mbox{\texttt{\mdseries\slshape K}} must be a finite field of square order. The points of $H(q)$ are either affine $P(a,b)$ satisfying $a^{q+1}=b^q+b$, or the infinite point \texttt{[ infinity ]}. One can use the \texttt{in} operation to test if a point lies on the Hermitian curve. }

 

\subsection{\textcolor{Chapter }{IndeterminatesOfHermitian{\textunderscore}Curve}}
\logpage{[ 3, 1, 3 ]}\nobreak
\hyperdef{L}{X7DC5282B7C7381C0}{}
{\noindent\textcolor{FuncColor}{$\triangleright$\enspace\texttt{IndeterminatesOfHermitian{\textunderscore}Curve({\mdseries\slshape Hq})\index{IndeterminatesOfHermitianuScoreCurve@\texttt{Indeterminates}\-\texttt{Of}\-\texttt{Hermitian{\textunderscore}}\-\texttt{Curve}}
\label{IndeterminatesOfHermitianuScoreCurve}
}\hfill{\scriptsize (function)}}\\


 returns the indeterminates of the function field of the Hermitian curve \mbox{\texttt{\mdseries\slshape C}}. }

 

\subsection{\textcolor{Chapter }{UnderlyingField}}
\logpage{[ 3, 1, 4 ]}\nobreak
\hyperdef{L}{X790470C48340E8F7}{}
{\noindent\textcolor{FuncColor}{$\triangleright$\enspace\texttt{UnderlyingField({\mdseries\slshape Hq})\index{UnderlyingField@\texttt{UnderlyingField}}
\label{UnderlyingField}
}\hfill{\scriptsize (attribute)}}\\


 The underlying field of a Hermitian curve is the field of coefficients of the
corresponding algebraic function field, it is a finite field of square order. }

 

\subsection{\textcolor{Chapter }{RandomPlaceOfGivenDegreeOfHermitian{\textunderscore}Curve}}
\logpage{[ 3, 1, 5 ]}\nobreak
\hyperdef{L}{X8095E4FA87ADE4EA}{}
{\noindent\textcolor{FuncColor}{$\triangleright$\enspace\texttt{RandomPlaceOfGivenDegreeOfHermitian{\textunderscore}Curve({\mdseries\slshape Hq, d})\index{RandomPlaceOfGivenDegreeOfHermitianuScoreCurve@\texttt{Random}\-\texttt{Place}\-\texttt{Of}\-\texttt{Given}\-\texttt{Degree}\-\texttt{Of}\-\texttt{Hermitian{\textunderscore}}\-\texttt{Curve}}
\label{RandomPlaceOfGivenDegreeOfHermitianuScoreCurve}
}\hfill{\scriptsize (operation)}}\\


 returns a random place of degree \mbox{\texttt{\mdseries\slshape d}} of the Hermitian curve \mbox{\texttt{\mdseries\slshape Hq}}, that is, a place defined over the field $GF(q^{2d})$. Notice that the place at infinity is has degree 1. }

 
\begin{Verbatim}[commandchars=!@|,fontsize=\small,frame=single,label=Example]
  !gapprompt@gap>| !gapinput@q:=5;|
  5
  !gapprompt@gap>| !gapinput@Y:=HermitianIndeterminates(GF(q^2),"Y1","Y2");|
  [ Y1, Y2 ]
  !gapprompt@gap>| !gapinput@Hq:=Hermitian_Curve(Y[1]);|
  <Hermitian curve over GF(25) with indeterminates [ Y1, Y2 ]>
  !gapprompt@gap>| !gapinput@|
  !gapprompt@gap>| !gapinput@UnderlyingField(Hq);|
  GF(5^2)
  !gapprompt@gap>| !gapinput@p:=RandomPlaceOfGivenDegreeOfHermitian_Curve(Hq,3);|
  <Hermitian place [ Z(5^6)^12002, Z(5^6)^14911 ] over indeterminates [ Y1, Y2 ]>
  !gapprompt@gap>| !gapinput@p in Hq;|
  true
  !gapprompt@gap>| !gapinput@[ infinity ] in Hq;|
  true
  !gapprompt@gap>| !gapinput@[0,0] in Hq;|
  false
  !gapprompt@gap>| !gapinput@Z(q)*[0,0] in Hq;|
  true
  !gapprompt@gap>| !gapinput@Size( AllRationalPlacesOfHermitian_Curve(Hq) );|
  126
\end{Verbatim}
 }

 
\section{\textcolor{Chapter }{Automorphisms of Hermitian curves}}\label{Automorphisms}
\logpage{[ 3, 2, 0 ]}
\hyperdef{L}{X7CCCCF377D442C10}{}
{
  

\subsection{\textcolor{Chapter }{FrobeniusAutomorphismOfHermitian{\textunderscore}Curve}}
\logpage{[ 3, 2, 1 ]}\nobreak
\hyperdef{L}{X84C46EC878997409}{}
{\noindent\textcolor{FuncColor}{$\triangleright$\enspace\texttt{FrobeniusAutomorphismOfHermitian{\textunderscore}Curve({\mdseries\slshape Hq})\index{FrobeniusAutomorphismOfHermitianuScoreCurve@\texttt{Frobenius}\-\texttt{Automorphism}\-\texttt{Of}\-\texttt{Hermitian{\textunderscore}}\-\texttt{Curve}}
\label{FrobeniusAutomorphismOfHermitianuScoreCurve}
}\hfill{\scriptsize (attribute)}}\\


 returns the Frobenius automorphism of the underlying field of the Hermitian
curve \mbox{\texttt{\mdseries\slshape Hq}}. More precisely, the output is an AC-Frobenius automorphism in the sense of
the package \textsf{OnAlgClosure}, acting on the algebraic closure of the underlying finite field. }

 

\subsection{\textcolor{Chapter }{IsHermitian{\textunderscore}CurveAutomorphism}}
\logpage{[ 3, 2, 2 ]}\nobreak
\hyperdef{L}{X7DBD2AA379E5928D}{}
{\noindent\textcolor{FuncColor}{$\triangleright$\enspace\texttt{IsHermitian{\textunderscore}CurveAutomorphism({\mdseries\slshape obj})\index{IsHermitianuScoreCurveAutomorphism@\texttt{IsHermitian{\textunderscore}}\-\texttt{Curve}\-\texttt{Automorphism}}
\label{IsHermitianuScoreCurveAutomorphism}
}\hfill{\scriptsize (Category)}}\\


 With automorphisms of an algebraic curve $C$ one means the automorphisms of the corresponding algebraic function field $K(C)$. For a Hermitian curve $H(q)$ over a finite field, $Aut(GF(q)(H(q)))$ is isomorphic to the projective general linear group $PGU(3,q)$. In particular, an automorphism of $H(q)$ can be represented by a $3\times 3$ unitary matrix over $GF(q^2)$. }

 

\subsection{\textcolor{Chapter }{Hermitian{\textunderscore}CurveAutomorphism}}
\logpage{[ 3, 2, 3 ]}\nobreak
\hyperdef{L}{X8553A238865EEB91}{}
{\noindent\textcolor{FuncColor}{$\triangleright$\enspace\texttt{Hermitian{\textunderscore}CurveAutomorphism({\mdseries\slshape Hq, mat})\index{HermitianuScoreCurveAutomorphism@\texttt{Hermitian{\textunderscore}}\-\texttt{Curve}\-\texttt{Automorphism}}
\label{HermitianuScoreCurveAutomorphism}
}\hfill{\scriptsize (operation)}}\\
\textbf{\indent Returns:\ }
 the automorphism of the Hermitian curve $H(q)$, given by the unitary matrix \mbox{\texttt{\mdseries\slshape mat}}. 

}

 
\subsection{\textcolor{Chapter }{AutomorphismGroup}}\logpage{[ 3, 2, 4 ]}
\hyperdef{L}{X87677B0787B4461A}{}
{
\noindent\textcolor{FuncColor}{$\triangleright$\enspace\texttt{UnitaryGroupToHermitian{\textunderscore}CurveAutGroup({\mdseries\slshape matgr, Hq})\index{UnitaryGroupToHermitianuScoreCurveAutGroup@\texttt{Unitary}\-\texttt{Group}\-\texttt{To}\-\texttt{Hermitian{\textunderscore}}\-\texttt{Curve}\-\texttt{Aut}\-\texttt{Group}}
\label{UnitaryGroupToHermitianuScoreCurveAutGroup}
}\hfill{\scriptsize (function)}}\\
\textbf{\indent Returns:\ }
 the group $G$ of automorphisms of the Hermitian curve \mbox{\texttt{\mdseries\slshape Hq}}, which correspond to the unitary group \mbox{\texttt{\mdseries\slshape matgr}}. 



 The permutation action of \mbox{\texttt{\mdseries\slshape matgr}} on the set of rational places of \mbox{\texttt{\mdseries\slshape Hq}} is stored as a nice monomorphism of \$G\$. \noindent\textcolor{FuncColor}{$\triangleright$\enspace\texttt{AutomorphismGroup({\mdseries\slshape Hq})\index{AutomorphismGroup@\texttt{AutomorphismGroup}}
\label{AutomorphismGroup}
}\hfill{\scriptsize (operation)}}\\
\textbf{\indent Returns:\ }
 the automorphism group of the Hermitian curve \mbox{\texttt{\mdseries\slshape Hq}}. The elements are Hermitian curve automorphisms. The group is isomorphic to $PPG(3,q)$, where $GF(q^2)$ is the underlying field of \mbox{\texttt{\mdseries\slshape Hq}}. 

}

 
\begin{Verbatim}[commandchars=!@|,fontsize=\small,frame=single,label=Example]
  !gapprompt@gap>| !gapinput@aut:=AutomorphismGroup(Hq);|
  <group of Hermitian curve automorphisms of size 378000>
  !gapprompt@gap>| !gapinput@Random(aut);|
  Hermitian_CurveAut([ [ Z(5)^0, Z(5^2)^11, Z(5^2)^19 ], [ Z(5^2)^20, Z(5^2)^16, Z(5^2)^21 ], [ Z(5), Z(5^2)^4, Z(5^2)^10 ] ])
\end{Verbatim}
 }

 
\section{\textcolor{Chapter }{Hermitian divisors}}\label{Divisors}
\logpage{[ 3, 3, 0 ]}
\hyperdef{L}{X7B79A0147BC8F34E}{}
{
  The following functions are available: 

\subsection{\textcolor{Chapter }{IsHermitian{\textunderscore}Divisor}}
\logpage{[ 3, 3, 1 ]}\nobreak
\hyperdef{L}{X7DC2715183F59DFB}{}
{\noindent\textcolor{FuncColor}{$\triangleright$\enspace\texttt{IsHermitian{\textunderscore}Divisor({\mdseries\slshape obj})\index{IsHermitianuScoreDivisor@\texttt{IsHermitian{\textunderscore}Divisor}}
\label{IsHermitianuScoreDivisor}
}\hfill{\scriptsize (Category)}}\\


 A Hermitian divisor is a divisor of an algebraic function field of the
Hermitian curve $H(q):X^{q+1}=Y^q+Y$. Hermitian divisors form an additive commutative group. }

 

\subsection{\textcolor{Chapter }{Hermitian{\textunderscore}DivisorConstruct}}
\logpage{[ 3, 3, 2 ]}\nobreak
\hyperdef{L}{X7C25E6A77FC66456}{}
{\noindent\textcolor{FuncColor}{$\triangleright$\enspace\texttt{Hermitian{\textunderscore}DivisorConstruct({\mdseries\slshape Hq, pts, ords})\index{HermitianuScoreDivisorConstruct@\texttt{Hermitian{\textunderscore}}\-\texttt{Divisor}\-\texttt{Construct}}
\label{HermitianuScoreDivisorConstruct}
}\hfill{\scriptsize (function)}}\\


 returns the Hermitian divisor over \mbox{\texttt{\mdseries\slshape Hq}} with points from \mbox{\texttt{\mdseries\slshape pts}} and corresponding orders from \mbox{\texttt{\mdseries\slshape ords}}. It checks the input. }

 

\subsection{\textcolor{Chapter }{Hermitian{\textunderscore}Divisor}}
\logpage{[ 3, 3, 3 ]}\nobreak
\hyperdef{L}{X8522DFC37BA9FE9B}{}
{\noindent\textcolor{FuncColor}{$\triangleright$\enspace\texttt{Hermitian{\textunderscore}Divisor({\mdseries\slshape Hq, pts, ords})\index{HermitianuScoreDivisor@\texttt{Hermitian{\textunderscore}Divisor}}
\label{HermitianuScoreDivisor}
}\hfill{\scriptsize (operation)}}\\
\noindent\textcolor{FuncColor}{$\triangleright$\enspace\texttt{Hermitian{\textunderscore}Divisor({\mdseries\slshape Hq, pairs})\index{HermitianuScoreDivisor@\texttt{Hermitian{\textunderscore}Divisor}}
\label{HermitianuScoreDivisor}
}\hfill{\scriptsize (operation)}}\\


 returns the corresponding Hermitian divisor over the Hermitian curve \mbox{\texttt{\mdseries\slshape Hq}}. The list \mbox{\texttt{\mdseries\slshape pts}} must be points of \mbox{\texttt{\mdseries\slshape Hq}}; the infinite point is \texttt{[ infinity ]}. The list \mbox{\texttt{\mdseries\slshape ords}} contains the respective orders. The elements of the list \mbox{\texttt{\mdseries\slshape pairs}} are the point-order pairs. }

 

\subsection{\textcolor{Chapter }{Hermitian{\textunderscore}Place}}
\logpage{[ 3, 3, 4 ]}\nobreak
\hyperdef{L}{X874B79D17AF7FD28}{}
{\noindent\textcolor{FuncColor}{$\triangleright$\enspace\texttt{Hermitian{\textunderscore}Place({\mdseries\slshape Hq, pt})\index{HermitianuScorePlace@\texttt{Hermitian{\textunderscore}Place}}
\label{HermitianuScorePlace}
}\hfill{\scriptsize (operation)}}\\


 returns the corresponding place of the Hermitian curve \mbox{\texttt{\mdseries\slshape Hq}}, where \mbox{\texttt{\mdseries\slshape pt}} is either an affine point \mbox{\texttt{\mdseries\slshape Hq}}, or the infinite point is \texttt{[ infinity ]}. }

 

\subsection{\textcolor{Chapter }{ZeroHermitian{\textunderscore}Divisor}}
\logpage{[ 3, 3, 5 ]}\nobreak
\hyperdef{L}{X7D106EB28504FFDD}{}
{\noindent\textcolor{FuncColor}{$\triangleright$\enspace\texttt{ZeroHermitian{\textunderscore}Divisor({\mdseries\slshape Hq})\index{ZeroHermitianuScoreDivisor@\texttt{Zero}\-\texttt{Hermitian{\textunderscore}}\-\texttt{Divisor}}
\label{ZeroHermitianuScoreDivisor}
}\hfill{\scriptsize (operation)}}\\


 returns the zero divisor over the Hermitian curve \mbox{\texttt{\mdseries\slshape Hq}}. }

 

\subsection{\textcolor{Chapter }{IsRationalHermitian{\textunderscore}Divisor}}
\logpage{[ 3, 3, 6 ]}\nobreak
\hyperdef{L}{X8634CC3B79155E29}{}
{\noindent\textcolor{FuncColor}{$\triangleright$\enspace\texttt{IsRationalHermitian{\textunderscore}Divisor({\mdseries\slshape D})\index{IsRationalHermitianuScoreDivisor@\texttt{IsRational}\-\texttt{Hermitian{\textunderscore}}\-\texttt{Divisor}}
\label{IsRationalHermitianuScoreDivisor}
}\hfill{\scriptsize (attribute)}}\\


 Returns true if \mbox{\texttt{\mdseries\slshape D}} is invariant under the Frobenius automorphism of the underlying Hermitian
curve. }

 

\subsection{\textcolor{Chapter }{UnderlyingField}}
\logpage{[ 3, 3, 7 ]}\nobreak
\hyperdef{L}{X790470C48340E8F7}{}
{\noindent\textcolor{FuncColor}{$\triangleright$\enspace\texttt{UnderlyingField({\mdseries\slshape D})\index{UnderlyingField@\texttt{UnderlyingField}}
\label{UnderlyingField}
}\hfill{\scriptsize (attribute)}}\\


 The underlying field of a Hermitian divisor is the field of coefficients of
the corresponding Hermitian curve. }

 

\subsection{\textcolor{Chapter }{Support}}
\logpage{[ 3, 3, 8 ]}\nobreak
\hyperdef{L}{X7B689C0284AC4296}{}
{\noindent\textcolor{FuncColor}{$\triangleright$\enspace\texttt{Support({\mdseries\slshape D})\index{Support@\texttt{Support}}
\label{Support}
}\hfill{\scriptsize (attribute)}}\\


 The support of a Hermitian divisor is the set of points with nonzero orders. }

 

\subsection{\textcolor{Chapter }{Valuation}}
\logpage{[ 3, 3, 9 ]}\nobreak
\hyperdef{L}{X80D67BB67A509A56}{}
{\noindent\textcolor{FuncColor}{$\triangleright$\enspace\texttt{Valuation({\mdseries\slshape D, pt})\index{Valuation@\texttt{Valuation}}
\label{Valuation}
}\hfill{\scriptsize (operation)}}\\


 The valuation of a Hermitian divisor \mbox{\texttt{\mdseries\slshape D}} at the point or place \mbox{\texttt{\mdseries\slshape pt}} is its corresponding order. }

 

\subsection{\textcolor{Chapter }{PrincipalHermitian{\textunderscore}Divisor}}
\logpage{[ 3, 3, 10 ]}\nobreak
\hyperdef{L}{X7DBEBF4283238ABF}{}
{\noindent\textcolor{FuncColor}{$\triangleright$\enspace\texttt{PrincipalHermitian{\textunderscore}Divisor({\mdseries\slshape Hq, f})\index{PrincipalHermitianuScoreDivisor@\texttt{Principal}\-\texttt{Hermitian{\textunderscore}}\-\texttt{Divisor}}
\label{PrincipalHermitianuScoreDivisor}
}\hfill{\scriptsize (operation)}}\\


 returns the principal divisor of the rational function \mbox{\texttt{\mdseries\slshape f}} of the Hermitian curve \mbox{\texttt{\mdseries\slshape Hq}}. }

 

\subsection{\textcolor{Chapter }{SupremumHermitian{\textunderscore}Divisor}}
\logpage{[ 3, 3, 11 ]}\nobreak
\hyperdef{L}{X85754A9B7AA29851}{}
{\noindent\textcolor{FuncColor}{$\triangleright$\enspace\texttt{SupremumHermitian{\textunderscore}Divisor({\mdseries\slshape D1, D2})\index{SupremumHermitianuScoreDivisor@\texttt{Supremum}\-\texttt{Hermitian{\textunderscore}}\-\texttt{Divisor}}
\label{SupremumHermitianuScoreDivisor}
}\hfill{\scriptsize (function)}}\\


 returns the place-wise maximum of the orders of \mbox{\texttt{\mdseries\slshape D1}} and \mbox{\texttt{\mdseries\slshape D2}}. }

 

\subsection{\textcolor{Chapter }{InfimumHermitian{\textunderscore}Divisor}}
\logpage{[ 3, 3, 12 ]}\nobreak
\hyperdef{L}{X7879FBF17B270989}{}
{\noindent\textcolor{FuncColor}{$\triangleright$\enspace\texttt{InfimumHermitian{\textunderscore}Divisor({\mdseries\slshape D1, D2})\index{InfimumHermitianuScoreDivisor@\texttt{Infimum}\-\texttt{Hermitian{\textunderscore}}\-\texttt{Divisor}}
\label{InfimumHermitianuScoreDivisor}
}\hfill{\scriptsize (function)}}\\


 returns the place-wise minimum of the orders of \mbox{\texttt{\mdseries\slshape D1}} and \mbox{\texttt{\mdseries\slshape D2}}. }

 

\subsection{\textcolor{Chapter }{PositivePartOfHermitian{\textunderscore}Divisor}}
\logpage{[ 3, 3, 13 ]}\nobreak
\hyperdef{L}{X7ACE607A7C753AC6}{}
{\noindent\textcolor{FuncColor}{$\triangleright$\enspace\texttt{PositivePartOfHermitian{\textunderscore}Divisor({\mdseries\slshape D})\index{PositivePartOfHermitianuScoreDivisor@\texttt{Positive}\-\texttt{Part}\-\texttt{Of}\-\texttt{Hermitian{\textunderscore}}\-\texttt{Divisor}}
\label{PositivePartOfHermitianuScoreDivisor}
}\hfill{\scriptsize (function)}}\\


 returns the positive part of the divisor \mbox{\texttt{\mdseries\slshape D}}. }

 

\subsection{\textcolor{Chapter }{NegativePartOfHermitian{\textunderscore}Divisor}}
\logpage{[ 3, 3, 14 ]}\nobreak
\hyperdef{L}{X78F4961F819285AC}{}
{\noindent\textcolor{FuncColor}{$\triangleright$\enspace\texttt{NegativePartOfHermitian{\textunderscore}Divisor({\mdseries\slshape D})\index{NegativePartOfHermitianuScoreDivisor@\texttt{Negative}\-\texttt{Part}\-\texttt{Of}\-\texttt{Hermitian{\textunderscore}}\-\texttt{Divisor}}
\label{NegativePartOfHermitianuScoreDivisor}
}\hfill{\scriptsize (function)}}\\


 returns the negative part of the divisor \mbox{\texttt{\mdseries\slshape D}}. }

 
\begin{Verbatim}[commandchars=!@|,fontsize=\small,frame=single,label=Example]
  !gapprompt@gap>| !gapinput@p_infty:=Hermitian_Place(Hq,[infinity]);|
  <Hermitian place [ infinity ] over indeterminates [ Y1, Y2 ]>
  !gapprompt@gap>| !gapinput@d:=3*p_infty-4*p;|
  <Hermitian divisor with support of length 2 over indeterminates [ Y1, Y2 ]>
  !gapprompt@gap>| !gapinput@Support(d);|
  [ [ infinity ], [ Z(5^6)^12002, Z(5^6)^14911 ] ]
  !gapprompt@gap>| !gapinput@UnderlyingField(d);|
  GF(5^2)
  !gapprompt@gap>| !gapinput@Zero(d);|
  <Hermitian divisor with support of length 0 over indeterminates [ Y1, Y2 ]>
  !gapprompt@gap>| !gapinput@Characteristic(d);|
  5
  !gapprompt@gap>| !gapinput@|
  !gapprompt@gap>| !gapinput@Valuation(d,p);|
  -4
  !gapprompt@gap>| !gapinput@Valuation(d,[1,2]);|
  0
  !gapprompt@gap>| !gapinput@|
  !gapprompt@gap>| !gapinput@fr:=FrobeniusAutomorphismOfHermitian_Curve(Hq);|
  AC_FrobeniusAutomorphism(5^2)
  !gapprompt@gap>| !gapinput@d^fr;|
  <Hermitian divisor with support of length 2 over indeterminates [ Y1, Y2 ]>
  !gapprompt@gap>| !gapinput@Support(d^fr);|
  [ [ infinity ], [ Z(5^6)^3194, Z(5^6)^13423 ] ]
  !gapprompt@gap>| !gapinput@Support(d);|
  [ [ infinity ], [ Z(5^6)^12002, Z(5^6)^14911 ] ]
\end{Verbatim}
 }

 
\section{\textcolor{Chapter }{Hermitian Riemann-Roch spaces}}\label{RRspaces}
\logpage{[ 3, 4, 0 ]}
\hyperdef{L}{X7DDB5BF585694D3F}{}
{
  

\subsection{\textcolor{Chapter }{Hermitian{\textunderscore}RiemannRochSpaceBasis}}
\logpage{[ 3, 4, 1 ]}\nobreak
\hyperdef{L}{X7B5089E879E3DBDD}{}
{\noindent\textcolor{FuncColor}{$\triangleright$\enspace\texttt{Hermitian{\textunderscore}RiemannRochSpaceBasis({\mdseries\slshape D})\index{HermitianuScoreRiemannRochSpaceBasis@\texttt{Hermitian{\textunderscore}}\-\texttt{Riemann}\-\texttt{Roch}\-\texttt{Space}\-\texttt{Basis}}
\label{HermitianuScoreRiemannRochSpaceBasis}
}\hfill{\scriptsize (function)}}\\


 returns a \textsc{basis} of the Riemann-Roch space of the Hermitian divisor \mbox{\texttt{\mdseries\slshape D}}, which is defined by $\{ f \in K[Y] \mid Div(f) \geq - D \}$. }

 
\begin{Verbatim}[commandchars=!@|,fontsize=\small,frame=single,label=Example]
  !gapprompt@gap>| !gapinput@a:=RandomPlaceOfGivenDegreeOfHermitian_Curve(Hq,3);|
  <Hermitian place [ Z(5^6)^5885, Z(5^6)^13071 ] over indeterminates [ Y1, Y2 ]>
  !gapprompt@gap>| !gapinput@fr:=FrobeniusAutomorphismOfHermitian_Curve(Hq);|
  AC_FrobeniusAutomorphism(5^2)
  !gapprompt@gap>| !gapinput@d:=Sum(AC_FrobeniusAutomorphismOrbit(fr,a));|
  <Hermitian divisor with support of length 3 over indeterminates [ Y1, Y2 ]>
  !gapprompt@gap>| !gapinput@IsRationalHermitian_Divisor(d);|
  true
  !gapprompt@gap>| !gapinput@|
  !gapprompt@gap>| !gapinput@bb:=Hermitian_RiemannRochSpaceBasis(5*d);|
  [ (Y1^3+Z(5^2)*Y1^2+Z(5^2)^17)/(Z(5^2)*Y1^3+Z(5^2)^17*Y1*Y2^2-Y2^3+Z(5^2)^14*Y1*Y2+Z(5^2)^11*Y2^2+Z(5^2)^11*Y1+Z(5^2)^7*Y2+Z(5^2)^\
  15), 
    (Z(5^2)^16*Y1^2+Z(5^2)^23*Y1+Y2+Z(5^2)^5)/(Z(5^2)*Y1^3+Z(5^2)^17*Y1*Y2^2-Y2^3+Z(5^2)^14*Y1*Y2+Z(5^2)^11*Y2^2+Z(5^2)^11*Y1+Z(5^2)\
  ^7*Y2+Z(5^2)^15), 
    (Z(5^2)^17*Y1^2+Y1*Y2+Z(5^2)^5*Y1+Z(5^2)^21)/(Z(5^2)*Y1^3+Z(5^2)^17*Y1*Y2^2-Y2^3+Z(5^2)^14*Y1*Y2+Z(5^2)^11*Y2^2+Z(5^2)^11*Y1+Z(5\
  ^2)^7*Y2+Z(5^2)^15), 
    (Y1^2*Y2+Z(5^2)^3*Y1^2+Z(5^2)^21*Y1+Z(5^2)^22)/(Z(5^2)*Y1^3+Z(5^2)^17*Y1*Y2^2-Y2^3+Z(5^2)^14*Y1*Y2+Z(5^2)^11*Y2^2+Z(5^2)^11*Y1+Z\
  (5^2)^7*Y2+Z(5^2)^15), 
    (Z(5^2)^23*Y1^2+Y2^2+Z(5^2)^8*Y1+Z(5^2)^13)/(Z(5^2)*Y1^3+Z(5^2)^17*Y1*Y2^2-Y2^3+Z(5^2)^14*Y1*Y2+Z(5^2)^11*Y2^2+Z(5^2)^11*Y1+Z(5^\
  2)^7*Y2+Z(5^2)^15), 
    (Y1*Y2^2+Z(5^2)^7*Y1^2+Z(5^2)^13*Y1+Z(5^2)^4)/(Z(5^2)*Y1^3+Z(5^2)^17*Y1*Y2^2-Y2^3+Z(5^2)^14*Y1*Y2+Z(5^2)^11*Y2^2+Z(5^2)^11*Y1+Z(\
  5^2)^7*Y2+Z(5^2)^15), 
    (Y2^3+Z(5^2)^5*Y1^2+Z(5^2)^11*Y1+Z(5^2)^14)/(Z(5^2)*Y1^3+Z(5^2)^17*Y1*Y2^2-Y2^3+Z(5^2)^14*Y1*Y2+Z(5^2)^11*Y2^2+Z(5^2)^11*Y1+Z(5^\
  2)^7*Y2+Z(5^2)^15) ]
  !gapprompt@gap>| !gapinput@Size(bb);|
  7
  !gapprompt@gap>| !gapinput@ForAll(bb,x->x=x^fr);|
  true
  !gapprompt@gap>| !gapinput@ForAll(bb,x->PrincipalHermitian_Divisor(Hq,x)>=-5*d);|
  true
\end{Verbatim}
 }

 
\section{\textcolor{Chapter }{Hermitian AG-codes}}\label{AGcodes}
\logpage{[ 3, 5, 0 ]}
\hyperdef{L}{X790A517B87412E1D}{}
{
  The following functions are available: 

\subsection{\textcolor{Chapter }{IsHermitian{\textunderscore}Code}}
\logpage{[ 3, 5, 1 ]}\nobreak
\hyperdef{L}{X7D67153387DB94E1}{}
{\noindent\textcolor{FuncColor}{$\triangleright$\enspace\texttt{IsHermitian{\textunderscore}Code({\mdseries\slshape obj})\index{IsHermitianuScoreCode@\texttt{IsHermitian{\textunderscore}Code}}
\label{IsHermitianuScoreCode}
}\hfill{\scriptsize (Category)}}\\
\noindent\textcolor{FuncColor}{$\triangleright$\enspace\texttt{IsHermitian{\textunderscore}FunctionalCode({\mdseries\slshape obj})\index{IsHermitianuScoreFunctionalCode@\texttt{IsHermitian{\textunderscore}}\-\texttt{Functional}\-\texttt{Code}}
\label{IsHermitianuScoreFunctionalCode}
}\hfill{\scriptsize (Category)}}\\
\noindent\textcolor{FuncColor}{$\triangleright$\enspace\texttt{IsHermitian{\textunderscore}DifferentialCode({\mdseries\slshape obj})\index{IsHermitianuScoreDifferentialCode@\texttt{IsHermitian{\textunderscore}}\-\texttt{Differential}\-\texttt{Code}}
\label{IsHermitianuScoreDifferentialCode}
}\hfill{\scriptsize (Category)}}\\


 A Hermitian code is an algebraic-geometric (AG) code defined on the Hermitian
curve of equation $X^{q+1}=Y^q+Y$. AG-codes are either of functional or of differential type. }

 

\subsection{\textcolor{Chapter }{GeneratorMatrixOfFunctionalHermitian{\textunderscore}CodeNC}}
\logpage{[ 3, 5, 2 ]}\nobreak
\hyperdef{L}{X8693655985BF0B1B}{}
{\noindent\textcolor{FuncColor}{$\triangleright$\enspace\texttt{GeneratorMatrixOfFunctionalHermitian{\textunderscore}CodeNC({\mdseries\slshape G, pls})\index{GeneratorMatrixOfFunctionalHermitianuScoreCodeNC@\texttt{Generator}\-\texttt{Matrix}\-\texttt{Of}\-\texttt{Functional}\-\texttt{Hermitian{\textunderscore}}\-\texttt{CodeNC}}
\label{GeneratorMatrixOfFunctionalHermitianuScoreCodeNC}
}\hfill{\scriptsize (function)}}\\


 returns the generator matrix of the functional AG code $C_L(D,G)$, where $D$ is the sum of the degree one places in the list \mbox{\texttt{\mdseries\slshape pls}}. The support of \mbox{\texttt{\mdseries\slshape G}} must be disjoint from \mbox{\texttt{\mdseries\slshape pls}}. }

 

\subsection{\textcolor{Chapter }{Hermitian{\textunderscore}FunctionalCode}}
\logpage{[ 3, 5, 3 ]}\nobreak
\hyperdef{L}{X877B39B9783ADE8B}{}
{\noindent\textcolor{FuncColor}{$\triangleright$\enspace\texttt{Hermitian{\textunderscore}FunctionalCode({\mdseries\slshape G, D})\index{HermitianuScoreFunctionalCode@\texttt{Hermitian{\textunderscore}}\-\texttt{Functional}\-\texttt{Code}}
\label{HermitianuScoreFunctionalCode}
}\hfill{\scriptsize (operation)}}\\
\noindent\textcolor{FuncColor}{$\triangleright$\enspace\texttt{Hermitian{\textunderscore}FunctionalCode({\mdseries\slshape G})\index{HermitianuScoreFunctionalCode@\texttt{Hermitian{\textunderscore}}\-\texttt{Functional}\-\texttt{Code}}
\label{HermitianuScoreFunctionalCode}
}\hfill{\scriptsize (operation)}}\\


 returns the functional AG code $C_L(D,G)=\{(f(P_1),\ldots,f(P_n)) \mid f\in L(G)\}.$ $D$ and $G$ are rational divisors of the Hermitian curve $H(q)$. $D=P_1+\cdots+P_n$, where $P_1,\ldots,P_n$ are degree one places of $H(q)$. The supports of $D$ and $G$ are disjoint. If $D$ is not given then it is the sum of affine rational places of $H(q)$, not contained in the support of $G$. By the Riemann-Roch theorem, functional codes have dimension at least $\deg(G)+1-g$, with equality if $\deg(G)>2g-2$. }

 

\subsection{\textcolor{Chapter }{Hermitian{\textunderscore}DifferentialCode}}
\logpage{[ 3, 5, 4 ]}\nobreak
\hyperdef{L}{X817FBF377D94ED1D}{}
{\noindent\textcolor{FuncColor}{$\triangleright$\enspace\texttt{Hermitian{\textunderscore}DifferentialCode({\mdseries\slshape G, D})\index{HermitianuScoreDifferentialCode@\texttt{Hermitian{\textunderscore}}\-\texttt{Differential}\-\texttt{Code}}
\label{HermitianuScoreDifferentialCode}
}\hfill{\scriptsize (operation)}}\\
\noindent\textcolor{FuncColor}{$\triangleright$\enspace\texttt{Hermitian{\textunderscore}DifferentialCode({\mdseries\slshape G})\index{HermitianuScoreDifferentialCode@\texttt{Hermitian{\textunderscore}}\-\texttt{Differential}\-\texttt{Code}}
\label{HermitianuScoreDifferentialCode}
}\hfill{\scriptsize (operation)}}\\


 returns the differential AG code $C_\Omega(D,G) = \{res_{P_1}(\omega),\ldots,res_{P_n}(\omega) \mid \omega \in
\Omega(G-D)\}.$ $D$ and $G$ are rational divisors of the Hermitian curve $H(q)$. $D=P_1+\cdots+P_n$, where $P_1,\ldots,P_n$ are degree one places of $H(q)$. The supports of $D$ and $G$ are disjoint. If $D$ is not given then it is the sum of affine rational places of $H(q)$, not contained in the support of $G$. By the Riemann-Roch theorem, functional codes have dimension $\deg(G)+1-g$. The differential code is the dual of the corresponding functional code. By
the Riemann-Roch theorem, differential codes have dimension at least $n-\deg(G)-1+g$, with equality if $\deg(G)>2g-2$. }

 

\subsection{\textcolor{Chapter }{Length}}
\logpage{[ 3, 5, 5 ]}\nobreak
\hyperdef{L}{X780769238600AFD1}{}
{\noindent\textcolor{FuncColor}{$\triangleright$\enspace\texttt{Length({\mdseries\slshape C})\index{Length@\texttt{Length}}
\label{Length}
}\hfill{\scriptsize (attribute)}}\\


 returns the length of the AG code \mbox{\texttt{\mdseries\slshape C}}. }

 

\subsection{\textcolor{Chapter }{GeneratorMatrixOfHermitian{\textunderscore}Code}}
\logpage{[ 3, 5, 6 ]}\nobreak
\hyperdef{L}{X839F5C6085C27928}{}
{\noindent\textcolor{FuncColor}{$\triangleright$\enspace\texttt{GeneratorMatrixOfHermitian{\textunderscore}Code({\mdseries\slshape C})\index{GeneratorMatrixOfHermitianuScoreCode@\texttt{Generator}\-\texttt{Matrix}\-\texttt{Of}\-\texttt{Hermitian{\textunderscore}}\-\texttt{Code}}
\label{GeneratorMatrixOfHermitianuScoreCode}
}\hfill{\scriptsize (attribute)}}\\


 returns the generator matrix of the AG code \mbox{\texttt{\mdseries\slshape C}} in \textsf{CVEC} matrix format. }

 

\subsection{\textcolor{Chapter }{DesignedMinimumDistance}}
\logpage{[ 3, 5, 7 ]}\nobreak
\hyperdef{L}{X84EB7DAF7DB9DB9F}{}
{\noindent\textcolor{FuncColor}{$\triangleright$\enspace\texttt{DesignedMinimumDistance({\mdseries\slshape C})\index{DesignedMinimumDistance@\texttt{DesignedMinimumDistance}}
\label{DesignedMinimumDistance}
}\hfill{\scriptsize (attribute)}}\\


 returns the designed minimum distance $\delta$ of the Hermitian AG code \mbox{\texttt{\mdseries\slshape C}}. When $\deg(G)\geq 2g-2$, then the general formulas for $\delta$ are as follows. For the functional code $C_L(D,G)$, $\delta=n-\deg(G)$, and for the differential code $C_\Omega(D,G)$, $\delta=\deg(G)-(2g-2)$. }

 
\begin{Verbatim}[commandchars=!@|,fontsize=\small,frame=single,label=Example]
  !gapprompt@gap>| !gapinput@Hermitian_FunctionalCode(5*d);|
  <[125,7] Hermitian AG-code over GF(5^2)>
  !gapprompt@gap>| !gapinput@a:=Sum(AllRationalAffinePlacesOfHermitian_Curve(Hq));|
  <Hermitian divisor with support of length 125 over indeterminates [ Y1, Y2 ]>
  !gapprompt@gap>| !gapinput@code:=Hermitian_FunctionalCode(8*d,a);|
  <[125,15] Hermitian AG-code over GF(5^2)>
  !gapprompt@gap>| !gapinput@Print(code);|
  Hermitian_FunctionalCode(Hermitian_Divisor(Hermitian_Curve(Y1),[ [ Z(5^6)^5885, Z(5^6)^13071 ], [ Z(5^6)^6485, Z(5^6)^13647 ], 
    [ Z(5^6)^6509, Z(5^6)^14295 ] ],[ 8, 8, 8 ]),Hermitian_Divisor(Hermitian_Curve(Y1),[ [ 0*Z(5), 0*Z(5) ], [ 0*Z(5), Z(5^2)^3 ], 
    [ 0*Z(5), Z(5^2)^9 ], [ 0*Z(5), Z(5^2)^15 ], [ 0*Z(5), Z(5^2)^21 ], [ Z(5)^0, Z(5)^3 ], [ Z(5)^0, Z(5^2) ], 
    [ Z(5)^0, Z(5^2)^4 ], [ Z(5)^0, Z(5^2)^5 ], [ Z(5)^0, Z(5^2)^20 ], [ Z(5), Z(5) ], [ Z(5), Z(5^2)^8 ], [ Z(5), Z(5^2)^13 ], 
    [ Z(5), Z(5^2)^16 ], [ Z(5), Z(5^2)^17 ], [ Z(5)^2, Z(5)^3 ], [ Z(5)^2, Z(5^2) ], [ Z(5)^2, Z(5^2)^4 ], [ Z(5)^2, Z(5^2)^5 ], 
    [ Z(5)^2, Z(5^2)^20 ], [ Z(5)^3, Z(5) ], [ Z(5)^3, Z(5^2)^8 ], [ Z(5)^3, Z(5^2)^13 ], [ Z(5)^3, Z(5^2)^16 ], 
    [ Z(5)^3, Z(5^2)^17 ], [ Z(5^2), Z(5)^0 ], [ Z(5^2), Z(5^2)^2 ], [ Z(5^2), Z(5^2)^7 ], [ Z(5^2), Z(5^2)^10 ], 
    [ Z(5^2), Z(5^2)^11 ], [ Z(5^2)^2, Z(5) ], [ Z(5^2)^2, Z(5^2)^8 ], [ Z(5^2)^2, Z(5^2)^13 ], [ Z(5^2)^2, Z(5^2)^16 ], 
    [ Z(5^2)^2, Z(5^2)^17 ], [ Z(5^2)^3, Z(5)^2 ], [ Z(5^2)^3, Z(5^2)^14 ], [ Z(5^2)^3, Z(5^2)^19 ], [ Z(5^2)^3, Z(5^2)^22 ], 
    [ Z(5^2)^3, Z(5^2)^23 ], [ Z(5^2)^4, Z(5)^3 ], [ Z(5^2)^4, Z(5^2) ], [ Z(5^2)^4, Z(5^2)^4 ], [ Z(5^2)^4, Z(5^2)^5 ], 
    [ Z(5^2)^4, Z(5^2)^20 ], [ Z(5^2)^5, Z(5)^0 ], [ Z(5^2)^5, Z(5^2)^2 ], [ Z(5^2)^5, Z(5^2)^7 ], [ Z(5^2)^5, Z(5^2)^10 ], 
    [ Z(5^2)^5, Z(5^2)^11 ], [ Z(5^2)^7, Z(5)^2 ], [ Z(5^2)^7, Z(5^2)^14 ], [ Z(5^2)^7, Z(5^2)^19 ], [ Z(5^2)^7, Z(5^2)^22 ], 
    [ Z(5^2)^7, Z(5^2)^23 ], [ Z(5^2)^8, Z(5)^3 ], [ Z(5^2)^8, Z(5^2) ], [ Z(5^2)^8, Z(5^2)^4 ], [ Z(5^2)^8, Z(5^2)^5 ], 
    [ Z(5^2)^8, Z(5^2)^20 ], [ Z(5^2)^9, Z(5)^0 ], [ Z(5^2)^9, Z(5^2)^2 ], [ Z(5^2)^9, Z(5^2)^7 ], [ Z(5^2)^9, Z(5^2)^10 ], 
    [ Z(5^2)^9, Z(5^2)^11 ], [ Z(5^2)^10, Z(5) ], [ Z(5^2)^10, Z(5^2)^8 ], [ Z(5^2)^10, Z(5^2)^13 ], [ Z(5^2)^10, Z(5^2)^16 ], 
    [ Z(5^2)^10, Z(5^2)^17 ], [ Z(5^2)^11, Z(5)^2 ], [ Z(5^2)^11, Z(5^2)^14 ], [ Z(5^2)^11, Z(5^2)^19 ], [ Z(5^2)^11, Z(5^2)^22 ], 
    [ Z(5^2)^11, Z(5^2)^23 ], [ Z(5^2)^13, Z(5)^0 ], [ Z(5^2)^13, Z(5^2)^2 ], [ Z(5^2)^13, Z(5^2)^7 ], [ Z(5^2)^13, Z(5^2)^10 ], 
    [ Z(5^2)^13, Z(5^2)^11 ], [ Z(5^2)^14, Z(5) ], [ Z(5^2)^14, Z(5^2)^8 ], [ Z(5^2)^14, Z(5^2)^13 ], [ Z(5^2)^14, Z(5^2)^16 ], 
    [ Z(5^2)^14, Z(5^2)^17 ], [ Z(5^2)^15, Z(5)^2 ], [ Z(5^2)^15, Z(5^2)^14 ], [ Z(5^2)^15, Z(5^2)^19 ], [ Z(5^2)^15, Z(5^2)^22 ], 
    [ Z(5^2)^15, Z(5^2)^23 ], [ Z(5^2)^16, Z(5)^3 ], [ Z(5^2)^16, Z(5^2) ], [ Z(5^2)^16, Z(5^2)^4 ], [ Z(5^2)^16, Z(5^2)^5 ], 
    [ Z(5^2)^16, Z(5^2)^20 ], [ Z(5^2)^17, Z(5)^0 ], [ Z(5^2)^17, Z(5^2)^2 ], [ Z(5^2)^17, Z(5^2)^7 ], [ Z(5^2)^17, Z(5^2)^10 ], 
    [ Z(5^2)^17, Z(5^2)^11 ], [ Z(5^2)^19, Z(5)^2 ], [ Z(5^2)^19, Z(5^2)^14 ], [ Z(5^2)^19, Z(5^2)^19 ], [ Z(5^2)^19, Z(5^2)^22 ], 
    [ Z(5^2)^19, Z(5^2)^23 ], [ Z(5^2)^20, Z(5)^3 ], [ Z(5^2)^20, Z(5^2) ], [ Z(5^2)^20, Z(5^2)^4 ], [ Z(5^2)^20, Z(5^2)^5 ], 
    [ Z(5^2)^20, Z(5^2)^20 ], [ Z(5^2)^21, Z(5)^0 ], [ Z(5^2)^21, Z(5^2)^2 ], [ Z(5^2)^21, Z(5^2)^7 ], [ Z(5^2)^21, Z(5^2)^10 ], 
    [ Z(5^2)^21, Z(5^2)^11 ], [ Z(5^2)^22, Z(5) ], [ Z(5^2)^22, Z(5^2)^8 ], [ Z(5^2)^22, Z(5^2)^13 ], [ Z(5^2)^22, Z(5^2)^16 ], 
    [ Z(5^2)^22, Z(5^2)^17 ], [ Z(5^2)^23, Z(5)^2 ], [ Z(5^2)^23, Z(5^2)^14 ], [ Z(5^2)^23, Z(5^2)^19 ], [ Z(5^2)^23, Z(5^2)^22 ], 
    [ Z(5^2)^23, Z(5^2)^23 ] ],[ 1, 1, 1, 1, 1, 1, 1, 1, 1, 1, 1, 1, 1, 1, 1, 1, 1, 1, 1, 1, 1, 1, 1, 1, 1, 1, 1, 1, 1, 
    1, 1, 1, 1, 1, 1, 1, 1, 1, 1, 1, 1, 1, 1, 1, 1, 1, 1, 1, 1, 1, 1, 1, 1, 1, 1, 1, 1, 1, 1, 1, 1, 1, 1, 1, 1, 1, 1, 
    1, 1, 1, 1, 1, 1, 1, 1, 1, 1, 1, 1, 1, 1, 1, 1, 1, 1, 1, 1, 1, 1, 1, 1, 1, 1, 1, 1, 1, 1, 1, 1, 1, 1, 1, 1, 1, 1, 
    1, 1, 1, 1, 1, 1, 1, 1, 1, 1, 1, 1, 1, 1, 1, 1, 1, 1, 1, 1 ]))
  !gapprompt@gap>| !gapinput@DesignedMinimumDistance(code);|
  101
  !gapprompt@gap>| !gapinput@LeftActingDomain(code);|
  GF(5^2)
  !gapprompt@gap>| !gapinput@UnderlyingField(code);|
  GF(5^2)
\end{Verbatim}
 

\subsection{\textcolor{Chapter }{Hermitian{\textunderscore}DecodeToCodeword}}
\logpage{[ 3, 5, 8 ]}\nobreak
\hyperdef{L}{X86ED0E6E799550B2}{}
{\noindent\textcolor{FuncColor}{$\triangleright$\enspace\texttt{Hermitian{\textunderscore}DecodeToCodeword({\mdseries\slshape C, w})\index{HermitianuScoreDecodeToCodeword@\texttt{Hermitian{\textunderscore}}\-\texttt{Decode}\-\texttt{To}\-\texttt{Codeword}}
\label{HermitianuScoreDecodeToCodeword}
}\hfill{\scriptsize (operation)}}\\


 Let $\delta$ be the designed minimum distance of \mbox{\texttt{\mdseries\slshape C}}, and define $t=[(\delta-1-g)/2]$. If there is a codeword $c\in C$ with $d(c,w)\leq t$ then $c$ is returned. Otherwise, the output is \texttt{fail}. 

The decoding algorithm is from [Hoholdt-Pellikaan 1995]. The function \texttt{Hermitian{\textunderscore}DECODER{\textunderscore}DATA} precomputes two matrices which are stored as attributes of the AG code. The
decoding consists of solving linear equations. }

 
\begin{Verbatim}[commandchars=@|B,fontsize=\small,frame=single,label=Example]
  @gapprompt|gap>B @gapinput|q:=4;B
  4
  @gapprompt|gap>B @gapinput|# construct the curve and the divisorsB
  @gapprompt|gap>B @gapinput|Y:=HermitianIndeterminates(GF(q^2),"Y1","Y2");B
  [ Y1, Y2 ]
  @gapprompt|gap>B @gapinput|Hq:=Hermitian_Curve(Y[1]);B
  <Hermitian curve over GF(16) with indeterminates [ Y1, Y2 ]>
  @gapprompt|gap>B @gapinput|P_infty:=Hermitian_Place(Hq,[infinity]); B
  <Hermitian place [ infinity ] over indeterminates [ Y1, Y2 ]>
  @gapprompt|gap>B @gapinput|B
  @gapprompt|gap>B @gapinput|fr:=FrobeniusAutomorphismOfHermitian_Curve(Hq);B
  AC_FrobeniusAutomorphism(2^4)
  @gapprompt|gap>B @gapinput|P4:=RandomPlaceOfGivenDegreeOfHermitian_Curve(Hq,5);B
  <Hermitian place [ Z(2,20)+Z(2,20)^3+Z(2,20)^4+Z(2,20)^6+Z(2,20)^11+Z(2,20)^13+Z(2,20)^15+Z(2,20)^17+Z(2,20)^18+Z(2,20)^19, 
    Z(2,20)+Z(2,20)^4+Z(2,20)^5+Z(2,20)^7+Z(2,20)^8+Z(2,20)^12+Z(2,20)^13+Z(2,20)^15+Z(2,20)^19 ] over indeterminates [ Y1, Y2 ]>
  @gapprompt|gap>B @gapinput|P4:=Sum(AC_FrobeniusAutomorphismOrbit(fr,P4));B
  <Hermitian divisor with support of length 5 over indeterminates [ Y1, Y2 ]>
  @gapprompt|gap>B @gapinput|G:=5*P4+7*P_infty;B
  <Hermitian divisor with support of length 6 over indeterminates [ Y1, Y2 ]>
  @gapprompt|gap>B @gapinput|Degree(G);B
  32
  @gapprompt|gap>B @gapinput|B
  @gapprompt|gap>B @gapinput|len:=50;B
  50
  @gapprompt|gap>B @gapinput|affpts:=AllRationalAffinePlacesOfHermitian_Curve(Hq);;B
  @gapprompt|gap>B @gapinput|D:=Sum(affpts{[1..len]});B
  <Hermitian divisor with support of length 50 over indeterminates [ Y1, Y2 ]>
  @gapprompt|gap>B @gapinput|B
  @gapprompt|gap>B @gapinput|# construct the AG differential codeB
  @gapprompt|gap>B @gapinput|Hermitian_DifferentialCode(G);B
  <[64,37] Hermitian AG-code over GF(2^4)>
  @gapprompt|gap>B @gapinput|agcode:=Hermitian_DifferentialCode(G,D);B
  <[50,23] Hermitian AG-code over GF(2^4)>
  @gapprompt|gap>B @gapinput|DesignedMinimumDistance(agcode);B
  22
  @gapprompt|gap>B @gapinput|Length(agcode)-Degree(G)-1;B
  17
  @gapprompt|gap>B @gapinput|B
  @gapprompt|gap>B @gapinput|# test codeword generationB
  @gapprompt|gap>B @gapinput|t:=Int((DesignedMinimumDistance(agcode)-1-Genus(G!.curve))/2);B
  7
  @gapprompt|gap>B @gapinput|sent:=Random(agcode);;B
  @gapprompt|gap>B @gapinput|err:=RandomVectorOfGivenWeight(GF(q),Length(agcode),t);;B
  @gapprompt|gap>B @gapinput|received:=sent+err;;B
  @gapprompt|gap>B @gapinput|B
  @gapprompt|gap>B @gapinput|# decodingB
  @gapprompt|gap>B @gapinput|sent_decoded:=Hermitian_DecodeToCodeword(agcode,received);B
  <cvec over GF(2,4) of length 50>
  @gapprompt|gap>B @gapinput|sent=sent_decoded;B
  true
\end{Verbatim}
 }

 
\section{\textcolor{Chapter }{Utilities for Hermitian AG-codes}}\label{Utilities}
\logpage{[ 3, 6, 0 ]}
\hyperdef{L}{X78C4CB0A87B9F316}{}
{
  

\subsection{\textcolor{Chapter }{RestrictVectorSpace}}
\logpage{[ 3, 6, 1 ]}\nobreak
\hyperdef{L}{X853BF1BC7F6462B8}{}
{\noindent\textcolor{FuncColor}{$\triangleright$\enspace\texttt{RestrictVectorSpace({\mdseries\slshape V, F})\index{RestrictVectorSpace@\texttt{RestrictVectorSpace}}
\label{RestrictVectorSpace}
}\hfill{\scriptsize (function)}}\\


 Let $K$ be a field and $V$ a linear subspace of $K^n$. The restriction of \mbox{\texttt{\mdseries\slshape V}} to the field \mbox{\texttt{\mdseries\slshape F}} is the intersection $V\cap F^n$. }

 

\subsection{\textcolor{Chapter }{UPolCoeffsToSmallFieldNC}}
\logpage{[ 3, 6, 2 ]}\nobreak
\hyperdef{L}{X8596810487A72298}{}
{\noindent\textcolor{FuncColor}{$\triangleright$\enspace\texttt{UPolCoeffsToSmallFieldNC({\mdseries\slshape f, q})\index{UPolCoeffsToSmallFieldNC@\texttt{UPolCoeffsToSmallFieldNC}}
\label{UPolCoeffsToSmallFieldNC}
}\hfill{\scriptsize (function)}}\\


 This non-checking function returns the same polynomial as \mbox{\texttt{\mdseries\slshape f}}, making sure that the coefficients are in $GF(q)$. }

 

\subsection{\textcolor{Chapter }{RandomVectorOfGivenWeight}}
\logpage{[ 3, 6, 3 ]}\nobreak
\hyperdef{L}{X7D6980057AF64327}{}
{\noindent\textcolor{FuncColor}{$\triangleright$\enspace\texttt{RandomVectorOfGivenWeight({\mdseries\slshape F, n, k})\index{RandomVectorOfGivenWeight@\texttt{RandomVectorOfGivenWeight}}
\label{RandomVectorOfGivenWeight}
}\hfill{\scriptsize (function)}}\\


 returns a random vector of $F^n$ of Hamming weight $k$. \noindent\textcolor{FuncColor}{$\triangleright$\enspace\texttt{RandomVectorOfGivenDensity({\mdseries\slshape F, n, delta})\index{RandomVectorOfGivenDensity@\texttt{RandomVectorOfGivenDensity}}
\label{RandomVectorOfGivenDensity}
}\hfill{\scriptsize (function)}}\\


 returns a random vector of $F^n$ in which the density of nonzero elements is approximatively $\delta$. \noindent\textcolor{FuncColor}{$\triangleright$\enspace\texttt{RandomBinaryVectorOfGivenWeight({\mdseries\slshape n, k})\index{RandomBinaryVectorOfGivenWeight@\texttt{RandomBinaryVectorOfGivenWeight}}
\label{RandomBinaryVectorOfGivenWeight}
}\hfill{\scriptsize (function)}}\\


 returns a random vector of $GF(2)^n$ of Hamming weight $k$. \noindent\textcolor{FuncColor}{$\triangleright$\enspace\texttt{RandomBinaryVectorOfGivenDensity({\mdseries\slshape n, delta})\index{RandomBinaryVectorOfGivenDensity@\texttt{RandomBinaryVectorOfGivenDensity}}
\label{RandomBinaryVectorOfGivenDensity}
}\hfill{\scriptsize (function)}}\\


 returns a random vector of $GF(2)^n$ in which the density of nonzero elements is approximatively $\delta$. }

 }

 }

        
\chapter{\textcolor{Chapter }{An example: BCH type codes as Hermitian AG codes (???)}}\label{HERmitian Example}
\logpage{[ 4, 0, 0 ]}
\hyperdef{L}{X79C9ACD17E1D249A}{}
{
   The following example constructs BCH type codes as Hermitian AG codes. 
\begin{Verbatim}[commandchars=!@|,fontsize=\small,frame=single,label=Example]
  !gapprompt@gap>| !gapinput@|
\end{Verbatim}
 }

    \def\bibname{References\logpage{[ "Bib", 0, 0 ]}
\hyperdef{L}{X7A6F98FD85F02BFE}{}
}

\bibliographystyle{alpha}
\bibliography{references}

\addcontentsline{toc}{chapter}{References}

\def\indexname{Index\logpage{[ "Ind", 0, 0 ]}
\hyperdef{L}{X83A0356F839C696F}{}
}

\cleardoublepage
\phantomsection
\addcontentsline{toc}{chapter}{Index}


\printindex

\newpage
\immediate\write\pagenrlog{["End"], \arabic{page}];}
\immediate\closeout\pagenrlog
\end{document}
